\documentclass{article}
\usepackage{sagetex}
\usepackage{pgffor}

\begin{document}



Using Sage\TeX, one can use Sage to compute things and put them into
your \LaTeX{} document. For example, there are
$\sage{number_of_partitions(1269)}$ integer partitions of $1269$.
You don't need to compute the number yourself, or even cut and paste
it from somewhere.

Here's some Sage code:

\begin{sageblock}
    f(x) = exp(x) * sin(2*x)
\end{sageblock}

The second derivative of $f$ is

\[
  \frac{\mathrm{d}^{2}}{\mathrm{d}x^{2}} \sage{f(x)} =
  \sage{diff(f, x, 2)(x)}.
\]

Here's a plot of $f$ from $-1$ to $1$:

\sageplot{plot(f, -1, 1)}

\begin{sagesilent}
	a = ZZ.random_element(-10,10)
	while a == 0:
	   a = ZZ.random_element(-10,10)
	b = ZZ.random_element(-10,10)
	c = ZZ.random_element(-10,10)
	poly = a*x^2 + b*x + c
\end{sagesilent}

\sage{a}
\sage{b}
\sage{c}

\sageplot{plot(poly) }


\begin{sagesilent}
	x = ZZ.random_element(-10,10)

	x = 3*x + 1

	f(x) = x^5
		
	q = 5
\end{sagesilent}


\[
\sage{diff(f, x, 1)(x)}
\]

\sage{q}


\begin{sagesilent}
	m = 0
	while m <= 5:
	   m = m + 1
\end{sagesilent}

\sage{m}

\begin{sageblock}
for i in range(5):
    p = (1+1)^i
\end{sageblock}

\sage{p}

\end{document}
